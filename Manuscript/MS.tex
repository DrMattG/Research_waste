\documentclass[]{article}
\usepackage{lmodern}
\usepackage{amssymb,amsmath}
\usepackage{ifxetex,ifluatex}
\usepackage{fixltx2e} % provides \textsubscript
\ifnum 0\ifxetex 1\fi\ifluatex 1\fi=0 % if pdftex
  \usepackage[T1]{fontenc}
  \usepackage[utf8]{inputenc}
\else % if luatex or xelatex
  \ifxetex
    \usepackage{mathspec}
  \else
    \usepackage{fontspec}
  \fi
  \defaultfontfeatures{Ligatures=TeX,Scale=MatchLowercase}
\fi
% use upquote if available, for straight quotes in verbatim environments
\IfFileExists{upquote.sty}{\usepackage{upquote}}{}
% use microtype if available
\IfFileExists{microtype.sty}{%
\usepackage{microtype}
\UseMicrotypeSet[protrusion]{basicmath} % disable protrusion for tt fonts
}{}
\usepackage[margin=1in]{geometry}
\usepackage{hyperref}
\hypersetup{unicode=true,
            pdftitle={Evidence accumulation for tackling research waste},
            pdfborder={0 0 0},
            breaklinks=true}
\urlstyle{same}  % don't use monospace font for urls
\usepackage{graphicx,grffile}
\makeatletter
\def\maxwidth{\ifdim\Gin@nat@width>\linewidth\linewidth\else\Gin@nat@width\fi}
\def\maxheight{\ifdim\Gin@nat@height>\textheight\textheight\else\Gin@nat@height\fi}
\makeatother
% Scale images if necessary, so that they will not overflow the page
% margins by default, and it is still possible to overwrite the defaults
% using explicit options in \includegraphics[width, height, ...]{}
\setkeys{Gin}{width=\maxwidth,height=\maxheight,keepaspectratio}
\IfFileExists{parskip.sty}{%
\usepackage{parskip}
}{% else
\setlength{\parindent}{0pt}
\setlength{\parskip}{6pt plus 2pt minus 1pt}
}
\setlength{\emergencystretch}{3em}  % prevent overfull lines
\providecommand{\tightlist}{%
  \setlength{\itemsep}{0pt}\setlength{\parskip}{0pt}}
\setcounter{secnumdepth}{0}
% Redefines (sub)paragraphs to behave more like sections
\ifx\paragraph\undefined\else
\let\oldparagraph\paragraph
\renewcommand{\paragraph}[1]{\oldparagraph{#1}\mbox{}}
\fi
\ifx\subparagraph\undefined\else
\let\oldsubparagraph\subparagraph
\renewcommand{\subparagraph}[1]{\oldsubparagraph{#1}\mbox{}}
\fi

%%% Use protect on footnotes to avoid problems with footnotes in titles
\let\rmarkdownfootnote\footnote%
\def\footnote{\protect\rmarkdownfootnote}

%%% Change title format to be more compact
\usepackage{titling}

% Create subtitle command for use in maketitle
\providecommand{\subtitle}[1]{
  \posttitle{
    \begin{center}\large#1\end{center}
    }
}

\setlength{\droptitle}{-2em}

  \title{Evidence accumulation for tackling research waste}
    \pretitle{\vspace{\droptitle}\centering\huge}
  \posttitle{\par}
    \author{}
    \preauthor{}\postauthor{}
    \date{}
    \predate{}\postdate{}
  
\usepackage{array}
\usepackage{caption}
\usepackage{graphicx}
\usepackage{siunitx}
\usepackage{colortbl}
\usepackage{multirow}
\usepackage{hhline}
\usepackage{calc}
\usepackage{tabularx}
\usepackage{threeparttable}
\usepackage{wrapfig}

\begin{document}
\maketitle

Matthew J. Grainger\textsuperscript{1}*, Frederike C.
Bolam\textsuperscript{2}, Gavin B. Stewart\textsuperscript{2}, Erlend B.
Nilsen\textsuperscript{1}

\textsuperscript{1}Norwegian Institute for Nature Research, P.O. Box
5685 Torgarden, 7485, Trondheim, Norway

\textsuperscript{2}Modelling Evidence and Policy Research Group, School
of Natural and Environmental Sciences, Ridley Building 2, Claremont
Road, Newcastle upon Tyne NE1 7RU, UK

*corresponding author: matthew.grainger@nina.no

\emph{There is an urgent need for a change in research workflows so that
pre-existing knowledge is better utilised in designing new research. A
formal assessment of the accumulated knowledge prior to research
approval would reduce the waste of already limited resources caused by
asking low priority questions.}

``Research waste'' is a well-established concept in medical
research\textsuperscript{1}. Research is wasted when its outcomes cannot
be used for the benefit of society\textsuperscript{2}, for example
because no new knowledge is gained or the knowledge gained cannot be
applied. Waste can occur at any of the four stages of the research
process\textsuperscript{2}; question setting; methods; accessibility;
and reporting (Table 1). In medicine, global research waste was
estimated in 2009 to cost US \$85 bn\textsuperscript{2}, with few signs
of improvement in the last decade\textsuperscript{1}. There is little
reason to believe that the situation is substantially different in
ecology and conservation, although there are no field-wide formal
assessments of research waste.

Emerging topics are beginning to address some of the factors that result
in wasted research efforts (Table 1). In particular, there is increased
focus on methodological improvements in individual studies
(e.g.~\textsuperscript{3},\textsuperscript{4}), and on open science
leading to improved accessibility and
reporting\textsuperscript{5},\textsuperscript{6}. Less formal effort is
devoted to the question setting stage. Here we suggest that ``Evidence
Synthesis'' should be considered an additional stage of research
(Table1, Figure S1). Evidence synthesis methods close the research
process into a loop, and will have additional benefits in terms of
reducing research waste at the question setting stage.

\hypertarget{reducing-waste-in-question-setting}{%
\section{Reducing waste in question
setting}\label{reducing-waste-in-question-setting}}

There are two related areas where research waste can be reduced by
taking into account the existing body of evidence by applying evidence
synthesis methods.

\hypertarget{low-priority-questions}{%
\subsection{Low priority questions}\label{low-priority-questions}}

New studies may ask low priority questions - those that are irrelevant
to stakeholders. The remedy to this is to include stakeholders in the
research commissioning process\textsuperscript{2}. Evidence synthesis,
or horizon scanning for novel problems, should be used to provide
evidence to practitioners, researchers and other stakeholders so that
they can identify research gaps that are important to them and to
develop future questions\textsuperscript{7}.

\hypertarget{the-answer-is-already-known-with-certainty}{%
\subsection{The answer is already known with
certainty}\label{the-answer-is-already-known-with-certainty}}

If a topic has been sufficiently addressed in the existing literature we
might already know the outcome with high certainty. Further studies that
fail to leverage this existing knowledge are at high risk of wasting
limited research resources. There are a variety of tools available for
research-funders and researchers to assess the state of knowledge on the
topic of interest. For example, systematic maps (also known as Evidence
gap maps or Evidence maps), were designed to give an overview of the
available evidence on a broad topic\textsuperscript{8}. They can
highlight where there is enough available evidence for a systematic
review or where primary research is required (i.e.~there is a lack of
evidence). Systematic reviews can be used to synthesise knowledge about
a narrow topic such as the evidence for the effectiveness of an
intervention and can provide a statistical summary of the pooled effect
size. The statistical combination of numerical data extracted from the
evidence base during the process of a systematic review is known as
meta-analysis. Meta-analysis is commonly used in conservation and
ecology6 providing an understanding of the magnitude of the known effect
of an intervention across individual studies. These results can then be
used to identify what a new research project can add to the current
evidence base.

\hypertarget{identifying-research-waste-with-cumulative-meta-analysis}{%
\section{Identifying research waste with cumulative
meta-analysis}\label{identifying-research-waste-with-cumulative-meta-analysis}}

In medicine, one additional tool used to quantify research waste is
cumulative meta-analysis. A cumulative meta-analysis typically describes
the accumulation of evidence (e.g., about the effectiveness of an
intervention) across time, and available estimates are added to the
analysis in chronological order\textsuperscript{9}. Using cumulative
meta-analysis, a researcher, funding agency or decision maker can
identify if there is sufficient evidence to be confident that a reported
effect is true. At this stage new trials are no longer required to
predict the outcome with satisfactory certainty and hence future
research waste will be avoided.

\hypertarget{an-applied-example}{%
\subsection{An applied example}\label{an-applied-example}}

As an example of the approach within an applied ecology situation, we
consider to what extent autonomous acoustic recorders can replace human
observers in wildlife sampling and monitoring when the focus is on
estimating species richness, which now has a long history in the
ecological literature\textsuperscript{10}. Technological advances over
the last two decades have allowed this potential to be explored fully,
and well over 150 field studies have sought to answer this question. A
meta-analysis in 2018\textsuperscript{10} explored the pooled effect of
these studies using a meta-analytic approach to estimate species
richness of birds. Based on the combined evidence from the included
studies, they concluded that when human observers (using point counts)
and sound recorders sample areas of equal size then there is no
difference between estimates of bird species richness. When properly
conducted (see specific advice in\textsuperscript{10}), it can be
inferred that sound recorders can be used to monitor aspects of
biodiversity as efficiently as human observers. Twenty-eight primary
studies published between 2000 and 2017 were included in the
meta-analysis. Taking the role as a research funder or researcher at the
question setting stage, we can utilise cumulative meta-analysis to
determine if we need another study quantifying the difference between
acoustic recorders and human observers for bird survey point counts. We
adapted the analysis of\textsuperscript{10} to demonstrate the use of
cumulative meta-analysis (see supplementary materials). The effect size
(i.e the magnitude of the difference between intervention and control)
of studies investigating the difference between autonomous acoustic
recorders and human observers in terms of bird species richness
estimates was consistently close to 0.07 since 2015 (Figure 1). This
means that there was no clear difference between acoustic recorders and
human observers on bird point counts. It would therefore be wasteful to
fund yet another study that addressed this specific question. To reduce
research waste we need to be able to first identify it. One option is to
use cumulative meta-analysis. The approach demonstrated here is well
known and tested in the medical literature and should not be challenging
to integrate into conservation and applied ecology workflows. Cumulative
meta-analysis has already been used in our field to assess time-lag
bias\textsuperscript{11} but is not commonly used in the way we have
shown here.

\hypertarget{caveats}{%
\subsection{Caveats}\label{caveats}}

There are several important caveats that need to be addressed. The
heterogeneity in reporting and the drive for novelty in publications
means that meta-analysis is currently challenging in applied ecology.
There might not be sufficient good quality research to quantify the
cumulative effect of even some apparently well studied phenomena.
Researchers are best placed to add to the evidence base by ensuring that
they use of comparable measures of outcomes rather than novel ones.\\
In addition, publication bias, where the direction of statistical
significance of the outcome influences the decision to publish the
result, might bias the evidence base available. This is a major caveat
for all evidence synthesis approaches, but one which can be identified.
With cumulative meta-analysis one can identify publication bias11 by
accumulating the effect sizes in order of journal impact factor for
example. Although this method makes it possible to detect publication
bias it will not not solve the underlying problem, and researchers
should endeavour to reduce publication bias by following open science
(Table 1). There may of course be a time lag between identifying that we
have sufficient information, and not conducting further research on a
topic, due to timescales of publications of original research as well as
associated meta-analyses. To address these and other issues around
evidence synthesis we have been exploring a systems modelling approach
to combine empirical evidence from systematic reviews and meta-analysis
with expert opinion which allows key areas of uncertainty in a topic to
be identified and prioritised for research focus
(e.g.~\textsuperscript{12}).

\hypertarget{outlook}{%
\subsection{Outlook}\label{outlook}}

Research waste can be reduced and it is the responsibility of funders as
well as individual researchers to do so. We agree with the statement
targeted at medicine 25 years ago that ``We need\ldots{}better research,
and research done for the right reasons''\textsuperscript{13}. Without a
change in focus ecology and conservation funding will continue to be
wasted which will be detrimental to our efforts to provide solutions to
global societal challenges.

\hypertarget{supplementary-information}{%
\section{Supplementary Information}\label{supplementary-information}}

\hypertarget{methods}{%
\subsection{Methods}\label{methods}}

We extracted the data and R code from the supplementary information
in\textsuperscript{10} to recreate their analysis. As such we are
dependant on the accurate extraction of data from the primary studies by
the original authors. We intended our re-analysis of their data to be an
example of the cumulative meta-analysis approach rather than to make
explicit recommendations about the use of acoustic recorders for avian
surveys. Building on their random effects meta-analysis we ran a
cumulative meta-analysis using the ``cumul'' function in the
``metafor''\textsuperscript{14} package in R. The cumulative
meta-analysis was ordered by publication year and plotted using the
ggplot2 package in R\textsuperscript{15} . Where a single study provided
more than one estimate of effect the order in which the estimates were
accumulated was the same as the order presented by\textsuperscript{10}
and treated as subsequent trials. Changing the order of that these
particular studies were accumulated made no difference to the stability
of the estimates over time (see Figure 3). The original code, the
original data, our additional code for running the analysis and plotting
can be found at \url{https://github.com/DrMattG/Research_waste}.

\hypertarget{references}{%
\section*{References}\label{references}}
\addcontentsline{toc}{section}{References}

\hypertarget{refs}{}
\leavevmode\hypertarget{ref-glasziou2018research}{}%
1. Glasziou, P. \& Chalmers, I. Research waste is still a scandal---an
essay by paul glasziou and iain chalmers. \emph{Bmj} \textbf{363,} k4645
(2018).

\leavevmode\hypertarget{ref-chalmers2009avoidable}{}%
2. Chalmers, I. \& Glasziou, P. Avoidable waste in the production and
reporting of research evidence. \emph{The Lancet} \textbf{374,} 86--89
(2009).

\leavevmode\hypertarget{ref-fraser2018questionable}{}%
3. Fraser, H., Parker, T., Nakagawa, S., Barnett, A. \& Fidler, F.
Questionable research practices in ecology and evolution. \emph{PloS
one} \textbf{13,} e0200303 (2018).

\leavevmode\hypertarget{ref-nilsen_bowler_linnell_2019}{}%
4. Nilsen, E. B., Bowler, D. \& Linnell, J. D. C. Exploratory and
confirmatory conservation research in the open science era. (2019).
doi:\href{https://doi.org/10.32942/osf.io/75a6f}{10.32942/osf.io/75a6f}

\leavevmode\hypertarget{ref-gurevitch2018meta}{}%
5. Gurevitch, J., Koricheva, J., Nakagawa, S. \& Stewart, G.
Meta-analysis and the science of research synthesis. \emph{Nature}
\textbf{555,} 175 (2018).

\leavevmode\hypertarget{ref-powers2019open}{}%
6. Powers, S. M. \& Hampton, S. E. Open science, reproducibility, and
transparency in ecology. \emph{Ecological applications} \textbf{29,}
e01822 (2019).

\leavevmode\hypertarget{ref-gold2013prioritizing}{}%
7. Gold, R. \emph{et al.} Prioritizing research needs based on a
systematic evidence review: A pilot process for engaging stakeholders.
\emph{Health Expectations} \textbf{16,} 338--350 (2013).

\leavevmode\hypertarget{ref-saran2018evidence}{}%
8. Saran, A. \& White, H. Evidence and gap maps: A comparison of
different approaches. \emph{Campbell Systematic Reviews} \textbf{14,}
1--38 (2018).

\leavevmode\hypertarget{ref-lau1992cumulative}{}%
9. Lau, J. \emph{et al.} Cumulative meta-analysis of therapeutic trials
for myocardial infarction. \emph{New England Journal of Medicine}
\textbf{327,} 248--254 (1992).

\leavevmode\hypertarget{ref-darras2018comparing}{}%
10. Darras, K. \emph{et al.} Comparing the sampling performance of sound
recorders versus point counts in bird surveys: A meta-analysis.
\emph{Journal of applied ecology} \textbf{55,} 2575--2586 (2018).

\leavevmode\hypertarget{ref-leimu2004cumulative}{}%
11. Leimu, R. \& Koricheva, J. Cumulative meta--analysis: A new tool for
detection of temporal trends and publication bias in ecology.
\emph{Proceedings of the Royal Society of London. Series B: Biological
Sciences} \textbf{271,} 1961--1966 (2004).

\leavevmode\hypertarget{ref-carrick2018planting}{}%
12. Carrick, J. \emph{et al.} Is planting trees the solution to reducing
flood risks? \emph{Journal of Flood Risk Management} (2018).

\leavevmode\hypertarget{ref-altman1994scandal}{}%
13. Altman, D. G. The scandal of poor medical research. (1994).

\leavevmode\hypertarget{ref-metafor}{}%
14. Viechtbauer, W. Conducting meta-analyses in R with the metafor
package. \emph{Journal of Statistical Software} \textbf{36,} 1--48
(2010).

\leavevmode\hypertarget{ref-ggplot2}{}%
15. Wickham, H. \emph{Ggplot2: Elegant graphics for data analysis}.
(Springer-Verlag New York, 2016).

\newpage

\begin{figure}
\centering
\includegraphics{MS_files/figure-latex/Add Trial-1.pdf}
\caption{Cumulative forest plot of the meta-analysis of Darras et
al.~(2018) on the difference between human observers and acoustic
recorders in terms of species richness.\label{forest}}
\end{figure}

\begin{figure}
\centering
\includegraphics{MS_files/figure-latex/flowc-1.pdf}
\caption{The production of research flows through five stages (blue
lines) all of which can lead to research waste\textsuperscript{2} (red
dashed lines). Ecology and conservation have begun to reduce waste by
focusing on methodological improvements and open science. Evidence
synthesis (including reporting to decision makers) can contribute to the
reduction in research waste by influencing question setting and
appropriate methods and design (black dashed lines).\label{flow}}
\end{figure}

\begin{figure}
\centering
\includegraphics{MS_files/figure-latex/Randomise order of the estimates in the same study-1.pdf}
\caption{Cumulative forest plot of the meta-analysis of Darras et
al.~(2018) on the difference between human observers and acoustic
recorders in terms of species richness. Estimates from the same study
inputted in random order\label{forestX}}
\end{figure}

\begin{verbatim}
## Warning: package 'huxtable' was built under R version 3.6.1
\end{verbatim}

\begin{verbatim}
## 
## Attaching package: 'huxtable'
\end{verbatim}

\begin{verbatim}
## The following object is masked from 'package:dplyr':
## 
##     add_rownames
\end{verbatim}

\begin{verbatim}
## The following object is masked from 'package:purrr':
## 
##     every
\end{verbatim}

\begin{verbatim}
## The following object is masked from 'package:scales':
## 
##     number_format
\end{verbatim}

\begin{verbatim}
## The following object is masked from 'package:ggplot2':
## 
##     theme_grey
\end{verbatim}

 
  \providecommand{\huxb}[2]{\arrayrulecolor[RGB]{#1}\global\arrayrulewidth=#2pt}
  \providecommand{\huxvb}[2]{\color[RGB]{#1}\vrule width #2pt}
  \providecommand{\huxtpad}[1]{\rule{0pt}{\baselineskip+#1}}
  \providecommand{\huxbpad}[1]{\rule[-#1]{0pt}{#1}}

\begin{table}[h]
\centering
\begin{threeparttable}
\begin{tabularx}{0.35\textwidth}{p{0.175\textwidth} p{0.175\textwidth}}


\hhline{}
\arrayrulecolor{black}

\multicolumn{1}{!{\huxvb{0, 0, 0}{0}}l!{\huxvb{0, 0, 0}{0}}}{\huxtpad{4pt}\raggedright \textbf{Employee}\huxbpad{4pt}} &
\multicolumn{1}{r!{\huxvb{0, 0, 0}{0}}}{\huxtpad{4pt}\raggedleft \textbf{Salary}\huxbpad{4pt}} \tabularnewline[-0.5pt]


\hhline{>{\huxb{0, 0, 0}{2}}->{\huxb{0, 0, 0}{2}}-}
\arrayrulecolor{black}

\multicolumn{1}{!{\huxvb{0, 0, 0}{0}}l!{\huxvb{0, 0, 0}{0}}}{\huxtpad{4pt}\raggedright John Smith\huxbpad{4pt}} &
\multicolumn{1}{r!{\huxvb{0, 0, 0}{0}}}{\huxtpad{4pt}\raggedleft 50000.00\huxbpad{4pt}} \tabularnewline[-0.5pt]


\hhline{}
\arrayrulecolor{black}

\multicolumn{1}{!{\huxvb{0, 0, 0}{0}}l!{\huxvb{0, 0, 0}{0}}}{\huxtpad{4pt}\raggedright Jane Doe\huxbpad{4pt}} &
\multicolumn{1}{r!{\huxvb{0, 0, 0}{0}}}{\huxtpad{4pt}\raggedleft 50000.00\huxbpad{4pt}} \tabularnewline[-0.5pt]


\hhline{}
\arrayrulecolor{black}

\multicolumn{1}{!{\huxvb{0, 0, 0}{0}}l!{\huxvb{0, 0, 0}{0}}}{\huxtpad{4pt}\raggedright David Hugh-Jones\huxbpad{4pt}} &
\multicolumn{1}{r!{\huxvb{0, 0, 0}{0}}}{\huxtpad{4pt}\raggedleft 40000.00\huxbpad{4pt}} \tabularnewline[-0.5pt]


\hhline{}
\arrayrulecolor{black}
\end{tabularx}\end{threeparttable}


\end{table}


\end{document}
